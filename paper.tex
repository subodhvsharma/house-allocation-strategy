\documentclass[a4paper,10pt]{article}
%\usepackage[utf8x]{inputenc}

%opening

\begin{document}
\title {A Dynamic Algorithm to Accelerate House Allocation Strategy at IITD}
\author {Subodh Sharma \and Rahul Garg}
\maketitle

\begin{abstract}
Housing allocation in IIT Delhi is a woefully slow procedure. In this
article, we present an algorithm which will potentially speed up the
allocation procedure without requiring changes to the rules of allocation
by the Institute. We additionally demonstrate that our algorithm is
{\em fair} and {\em stable} (i.e., every allottee gets a unit that is
scored more than his/her current allotment) in its allocation.
\end{abstract}

\section{Introduction}
Housing allocation procedure at IIT Delhi suffers from the ills
resulting from the combination of sequential processing and
non-automation.  This often results in inordinate delays in the
notification of new available units for accommodation. Most notably,
those faculty members who are yet to be allocated faculty housing are
the worst sufferers of these systemic delays. 
%

Briefly, the current housing allocation procedure crawls in the
following manner: (i) assuming a unit is empty, the staff of the
estate office first performs a fitness check of the unit; this may
take anywhere from 2 to 4 weeks of time (depending on the workload on the 
estate office), (ii) subsequently, the estate
office issues a notification to faculty members indicating that the
house is up for grabs and is accepting applications from the faculty
members indicating their expression of interest (EoI) in the unit; the
deadline is typically set 2 weeks from the date of issue of the
notification, (iii) after which the allotment to a faculty who is
highest in {\em seniority} takes place (the rules of assigning
seniority are fixed and provided by the Institute), (iv) the mandatory 
basic cleaning and whitewash is performed by the estate office, 
which typically takes 4-8 weeks and (v) finally the allottee moves
with 1-2 weeks of the cleaning and the white-wash. The total time from 
notification to the movement of the allottee can take anywhere from 10-15 weeks. 

%
In particular, the problem for new faculty members exacerbates when
senior faculty members who have already been allotted units also
compete for the new unit.
%
In addition to the above problem, the afore-mentioned process of
allotment is severely sequential and we believe that there is
parallelism that can be exploited similar to instruction-level
parallelism in computer processors. Notice that after step (iii),
procedure of allotment of the unit, which is to be vacated soon by the
allottee, can be initiated.





\begin{thebibliography}{9}
\bibitem{stable-marriage-problem}
D. GALE and S. SHAPLEY, 
College Admissions and the Stability of Marriage. 

\end{thebibliography}




\end{document}
